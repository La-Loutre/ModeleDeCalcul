\documentclass[base.tex]{subfiles}
\begin{document}
\[decidable\Rightarrow semi-decidable\]
A peu près tous les ensembles que l'on peut nommer sont décidables.

\textbf{Exemples d'ensembles décidables :}
\begin{itemize}
\item Nombres pairs
\item Nombres premiers
  \item Les mots en alphabet ASCII qui représentent des programmes C.
\end{itemize}

\textbf{Exemple d'ensemble semi-décidable qui n'est pas décidable :}\\
L'ensemble des paires $(p,x) , p \in \mathbb{P} , x \in \mathbb{N}$ , telles que $p$ s'arrête sur $x$ ($p$ appliquer à $x$ s'arrête ).\\
\\
\textbf{Pas décidable :}\\
L'indécidabilité du problème de l'arrêt .
\\
Pourquoi cet ensemble est-il semi-décidable ? \\
On a une instance $(p,x)$ , on veut savoir si $p$ s'arrête sr $x$ , et si c'est le cas , répondre oui.\\
Comment agir ? Lancer $p$ sur $x$ et attendre . Quand ( et si ) $p$ s'arrête , on répond oui .


\subsubsection{Union/intersection/complémentaire}
\begin{itemize}
\item \textbf{L'union} de deux ensembles décidable est décidable \\
  \emph{Preuve :} soient deux ensemble A et B . Répondre "oui" si au moins un des deux algorithmes répond "oui" , et répondre "non" si tous les deux ont répondu "non".
\item \textbf{L'union} de deux ensemble semi-décidables est semi-décidable.\\
  \emph{Preuve :} la précédente, terminée avant la partie "et répondre non"\\
  \emph{Subtilité :}Il faut lancer deux programmes en parallèle pour que le deuxième n'attende pas la réponse du premier.
\item \textbf{L'intersection} des deux ensembles (semi-)décidables est (semi-)décidable. (vrai pour décidable et semi-décidable).\\
  Il faut avoir 2 réponses positives.
\item \textbf{Le complément} d'un ensemble décidable est décidable.\\
  \emph{Preuve :} Interchanger "oui" et "non"
\item Pour les ensembles semi-décidables en général ce n'est pas vrai.
\end{itemize}
\medskip
\textbf{Proposition} : Si un ensemble et son compléments sont tout les deux semi-décidables alors ils sont tous les deux décidables.\\
\emph{Preuve} : Supposons que $\mathbb{A}$ et $\bar{\mathbb{A}}$ soient semi-décidables . Il existe donc un algorithme qui répond positivement si $x \in \mathbb{A}$ ,et un autre qui répond positivement si $x\in \bar{\mathbb{A}}$ (c'est a dire si $x \not\in \mathbb{A}$). Il faut les lancer tous les deux en aprallèle , et si c'est le premier ($\mathbb{A}$) qui à répondu "oui" on répond "oui" , et si c'est le deuxième ($\bar{\mathbb{A}}$) on répond "non".
\subsubsection{Algorithmes énumératifs }% subsubsection ou subsectio ???? feuille1
Un tel algorithme
\begin{itemize}
\item Ne prend rien en entrée
\item ne s'arrête jamais
\item Il sort de temps à autre des éléments d'un ensemble ( des entiers ou des objets d'autre nature).
\end{itemize}
\medskip
Un tel algorithme génére , ou énumère un ensemble .
\\
\\
\textbf{Exemple :} 
\begin{verbatim}
n=0
tantque n >= 0 faire:
   m=n^2
   sortir m
   n = n+1
fintantque
\end{verbatim}
\medskip
\textbf{Proposition :}
\begin{itemize}
\item L'ensemble des résultats d'un algorithme énumératif est semi-décidable.
\item Tout ensemble semi-décidable peut être crée par un algorithme énumératif
\end{itemize}
\medskip
La première proposition est vue en TD , pour la deuxième :\\
Soit un ensemble $\mathbb{A}$ semi-décidable , et soit un algorithme $P$ qui répond par "oui" si $x \in \mathbb{A}$ .\\
Notre objectif : énumérer (c'est à dire sortir au fur et à mesure ) tout les éléments de $\mathbb{A}$. Comment faire ? Il faut lancer l'algorithme $P$ sur toutes les instance $x$ en parallèle.\\ %voir feuille 1 notes
Dès qu'il y a une réponse positive pour un certain $x$ , il faut le sortir en tant que résultat de l'algorithme énumératif .
\medskip
\textbf{Tout ensemble fini est décidable}
\medskip
\textbf{Remarque :} Nous avons parlé des intersections et des unions de deux ensembles décidables et semi-décidables. Les même propriétés restent vrais pour \textit{des familles finies d'ensemble}.
\\
\\
C'est pas forcément vrai pour les unions et intersections en quantité infinies.N'importe quel sous ensemble de $\mathbb{N}$ peut être obtenu comme union des singleton (ensemble ne comportant qu'un seul points ) .

\section{Complexité de Kolmogorov}
Il s'agit de complexité de description\\
Oublier temporairement le problème d'efficacité des algorithmes . On travail dans l'univers des mots binaires $\{0,1\}^*$ que l'on identifie avec des entiers positifs 1,2,3,..., de la manière suivante :
%TODO
%schéma a mettre ( schéma 1 feuille 2)
\\
\\
\textbf{La taille de l'écriture }
\begin{itemize}
\item Si on se représente un mot binaire , c'est la longueur de ce mot .
  \item Si on se représente le même objet comme un entier $m$ c'est $\lfloor \log _2 m \rfloor\simeq \log m$
\end{itemize}
\medskip
Soit un algorithme $f=\{0,1\}^* \rightarrow \{0,1\}^*$ \\
La complexité de mot $x \in \{0,1\}^*$ selon $f$ est : %voir feuille 2
Si f(t) = x
\[K_f(x)=min|t|\]
et si il tel $t$t n'existe pas 
\[K_f(x)=\infty\]
On dit que $t$ est une description de $x$ selon $f$ , et l'algorithme $f$ construit $x$ à partir de cette description.\\
\\
Un gros défaut : la complexité n'est pas "objective": elle dépend d'un algorithme $f$ choisit.\\
\\
\textbf{Théorème (Kolmogorov ,1965)}\\
Il existe un algorithme $f_0$ , dit optimal , tel que , quel que soit un autre algorithme $f$ on a :
\[K_{f_0}(x) \leq K_f(x) +C\]
La constante $C$ dépend de $f$.
\\
Donc on choisit , une fois pour toutes , un algorithme optimal $f_0$, et on définit
\[K(x) =K_{f_0}(x)\]
On peut avoir deux algorithmes optimaux $f_1$ et $f_2$ , mais alors
\[|K_{f_1}(x)-K_{f_2}(x)| \leq C\]
\end{document}
