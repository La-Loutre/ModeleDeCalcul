\documentclass[base.tex]{subfiles}
\begin{document}
\section{Calculabilité}
Ce qui est calculable et ce qui ne l'est pas
\\
Comment travailler avec des ensembles infinis :
\begin{itemize}
\item Bijections
\item Coordinalité
\item Dénombrables et non dénombrables
\item Diagonalisation
\item Fonction calculable et non calculable
\item Ensemble décidable et non décidables
\end{itemize}

\textbf{Machine de turing}
Automate avec une mémoire externe non limitée

\subsection{Exemple concrets de problèmes indécidables}
\textbf{Complexité}
\begin{itemize}
\item Réduction polynomiale
\item classe NP
\item NP complétude
\item Théorème de COOK-Lenon
\end{itemize}
Problème "p=NP?"
\\
\\
Comment mesurer la taille des ensembles infinis ?\\
Nombre (entier non négatif) : une abstraction.
\\
\\
\textbf{Définition}\\
On dit que 2 ensemble x et y sont de même cardinalité s'il existe une bijection
\[f:X->Y\]
\\
\\
\textbf{Bijection entre X et Y}:
Une fonction $f: X -> Y$ telle que :
\begin{itemize}
\item Pour tout $x \in X$ il existe un et un seul $y \in Y$ tel que $y=f(x)$
\item Pour tout $y \in Y$ il existe un et un seul $x \in X$ tel que $f(x) = y$
\end{itemize}
C'est à dire , il existe une fonction inverse \[x = f^{-1}(y)\]
%Schéma a mettre ici
\textbf{Définition}\\
Un ensemble $X$ est dénombrable s'il existe une bijection entre $X$ et $\mathbb{N}$
\\
\\
\textbf{Intuitivement} : Un ensemble est dénombrable s'il est infini et si on peut numéroter ses éléments par $0,1,2,...$
\subsection{Exemple d'ensemble dénombrables (voir en TD)}
\begin{itemize}
\item Carrés ${0,1,4,9,16,25,...}$
\item $\mathbb{Z} = \{0,-1,1,-2,2,...\}$
\item $\mathbb{N}^{2}=\{(x,y)|x,y \in \mathbb{N}\}$
\item Nombres rationnels = $\mathbb{Q}$ (fractions)
\item $\mathbb{A}^*$ ou $\mathbb{A}$ est un alphabet fini
\item Ensemble des graphes finis
\item Programme C
\end{itemize}
Soit l'ensemble $\mathbb{F}$ des fonctions $f\mathbb{N}$ .
$\mathbb{F} = \{f|f$ une fontion de $\mathbb{N}$ vers $\mathbb{N}$
\\
\textbf{Proposition}
\\
L'ensemble $\mathbb{F}$ n'est pas dénombrable .Méthode de preuve : \underline{diagonalisation}
\\
Mettre en bijection avec $\mathbb{N} \Leftrightarrow $  numéroté.\\
Supposons que l'ensemble de toutes les fonctions puisse être numéroté.
\[f0,f1,f2,f3,...\]
et cette liste contient toutes les fonctions $f:\mathbb{N}\rightarrow\mathbb{N}$
$\begin{array}{l|cr}xf & 0 & 1\\ \hline f0 & 0 & 0\\ f1 & 3 & 2\\\end{array}$
  \\
  On suppose que dans les lignes de cette table se trouvent toute les fonctions , et on va fabriquer une autre fonction, qui n'est pas dans la table.
  %tableau
\\
Cette fonction ne coincide pas avec f0 , ni avec f1 , ni avec f2 , etc .
Elle ne figure pas dans notre table !
\\
\\
\textbf{Reformulation}\\
L'ensemble des suites binaires infinis n'est pas dénombrable.\\
L'ensemble $P(\mathbb{N})$ des parties de $\mathbb{N}$ n'est pas dénombrable.\\
L'ensemble des points du segment [0,1] n'est pas dénombrable. Par conséquent $\mathbb{R}$ aussi .(point $t\in [0,1] \rightarrow$ son développement binaire : $t=0,101001...$
\\
\\
\subsection{Preuve par diagonalisation}
\textbf{George Cantor} (1874,publié 1891)
\\
Les années 1930:
\begin{itemize}
\item L'ensemble des algorithmes ( des programmes ) est dénombrable.
\item L'ensemble des fonctions $f:\mathbb{N}\rightarrow\mathbb{N}$ n'est pas dénombrable.
\item L'ensemble des fonctions est plus grand que l'ensemble des algorithmes , (il y a des fonctions pour lesquelles il n'y a pas d'algorithme pour les calculer.
\end{itemize}
Le premier objectif du cours : \\
Etudier le phénomène de non-calculabilité.\\
\\
\underline{Un exemple :} David Hilbert(1901):\\
23 problèmes pour le 20ème siècle .
\\
\\
\subsubsection{Le problème numéro 10}
Trouver un algorithme pour le problème suivant:\\
\underline{Instance :} Une équation $P(x1,...,xn) = 0$ où $P$ est un polynôme à coefficients entiers. \\
\underline{Question :} Existe ti'l des solutions en nombre entiers ?\\
\underline{Exemple :}
\begin{itemize}
\item $x^2 - 2y^2 = 1$ une instance de solution.
\item $x^n+y^n=z^n$ Pour $n=2$ : infinité de solution ,$n\geq3$ pas de solution.
  \item $y^2 = x(x-1)(x-a)$ Il existe un algorithme (la réponse dépend de a)
\end{itemize}
1971 Yuri Matigasench
Un tel algorithme n'existe pas !

\subsubsection{Théoreme de Cantor-Bernstein}
Soit deux ensemble $\mathbb{A}$ et $\mathbb{B}$ .
Si il existe une bijection $f$ entre $\mathbb{A}$ et un sous-ensemble $\mathbb{B}_1\subseteq\mathbb{B}$ , et aussi une bijection $g$ entre $\mathbb{B}$ et un sous ensemble $\mathbb{A}_1\subseteq\mathbb{A}$ alors il existe une bijection entre $\mathbb{A}$ et $\mathbb{B}$ .
\[f:\mathbb{A}\rightarrow\mathbb{B}\]
\[f(\mathbb{A})=\mathbb{B}_1\subseteq\mathbb{B}\]
\[g:\mathbb{B}\rightarrow\mathbb{A}\]
\[g(\mathbb{B})=\mathbb{A}_1\subseteq\mathbb{A}\]
%schéma feuille 2


\end{document}
